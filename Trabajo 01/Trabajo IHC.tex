\documentclass[11pt]{beamer}
\usepackage{listings} % Include the listings-package
\usepackage[T1]{fontenc}
\usepackage[utf8]{inputenc}
\usepackage[english]{babel}
\usepackage{amsmath}
\usepackage{amssymb, amsfonts, latexsym, cancel}
\usepackage{float}
\usepackage{graphicx}
\usepackage{epstopdf}
\usepackage{subfigure}
\usepackage{hyperref}
\usepackage{indentfirst}
%\usepackage{authblk}
\usepackage{blindtext}
\usepackage{booktabs} % Allows the use of \toprule, 
\usepackage{filecontents}
\usepackage{courier} %% Sets font for listing as Courier.
\usepackage{listings}
\usepackage{ragged2e}
\usepackage{listings, xcolor}
%\usepackage{parskip}
%\usepackage[margin=10cm]{geometry}
\lstset{
tabsize = 2, %% set tab space width
showstringspaces = false, %% prevent space marking in strings, string is defined as the text that is generally printed directly to the console
numbers = left, %% display line numbers on the left
commentstyle = \color{green}, %% set comment color
keywordstyle = \color{blue}, %% set keyword color
stringstyle = \color{red}, %% set string color
rulecolor = \color{black}, %% set frame color to avoid being affected by text color
basicstyle = \small \ttfamily , %% set listing font and size
breaklines = true, %% enable line breaking
numberstyle = \tiny,
}
\usepackage{caption}
\DeclareCaptionFont{white}{\color{white}}
\DeclareCaptionFormat{listing}{\colorbox{gray}{\parbox{\textwidth}{#1#2#3}}}
\captionsetup[lstlisting]{format=listing,labelfont=white,textfont=white}
\definecolor{urlColor}{rgb}{0.06, 0.3, 0.57}
\definecolor{linkColor}{rgb}{0.57, 0.0, 0.04}
\definecolor{fileColor}{rgb}{0.0, 0.26, 0.26}
\hypersetup{
    colorlinks=true,
    linkcolor=linkColor,
    filecolor=fileColor,      
    urlcolor=urlColor,
}
\urlstyle{same}
\setbeamercovered{transparent}
%\usetheme{Boadilla}
\usetheme{CambridgeUS}
%\usetheme{Berkeley}
%\usetheme{Warsaw}
%\usetheme{Madrid}

\title[Introducción]{\bf\Huge Pautas de diseño de la interfaz de usuario}
\subtitle{Interacción Humano Computador}

\author[ddiazti]
{
	Deyson Victor Diaz Ticona \inst{1}
	Kevin Yunior Ccorimanya Paucar \inst{2}
	Denilson Flores Valdivia \inst{3}
	Marco Antonio Ponce de Leon \inst{4}
}
\institute[UNSA]
{
\inst{1}
\inst{2}
\inst{3}
\inst{4}
System Engineering School\\
System Engineering and Informatic Department\\
Production and Services Faculty\\
San Agustin National University of Arequipa
}

\date[2020-09-15]{\scriptsize{2020-09-15}}
%\logo{\includegraphics[width=3.0cm]{logo_unsa.jpg}}
\titlegraphic{\includegraphics[width=1.0cm]{logo_unsa.jpg}}

\begin{document}

\begin{frame}
\titlepage
\end{frame}

\begin{frame}
\frametitle{Content}
\tableofcontents
\end{frame}

\section{Pautas de diseño de la interfaz de usuario}
\begin{frame}

\frametitle{Pautas de diseño de la interfaz de usuario}
\begin{itemize}
\item Sistemas informáticos interactivos
\item Pautas de diseño de la interfaz de usuario. (Reglas de diseño):
\item Cheriton (1976) - Tiempo compartido.
\item Norman (1983) - Cognición humana.
\item Smith y Mosier (1986)
\item Shneiderman (1987) - Ocho reglas de oro.
\item Brown (1988).
\item Nielsen y Molich (1990) - Evaluación heurística.
\item Nielsen y Mack (1994).
\item Stone et al. (2005).
\item Koyani y col. (2006).
\item {\bf Johnson (2007).}
\item Shneiderman y Plaisant (2009).
\item Microsoft, Apple Computer y Oracle - 2009, 2009, 2001
\end{itemize}
\end{frame}

\begin{frame}
\frametitle{Johnson (2007)}
\begin{itemize}
\color{red}
\item Análisis de Tareas
    \par
    \justify
    \color{black}
    Describir en detalle cómo analizar los objetivos y tareas de los usuarios. Por ahora es suficiente decir que un buen análisis de tareas responde a estas preguntas:
    \begin{itemize}
    \item ¿Qué objetivos quieren alcanzar los usuarios al utilizar la aplicación?
    \item ¿Qué conjunto de tareas humanas pretende admitir la aplicación?
    \item ¿Qué tareas son más importantes y cuáles son las menos importantes?
    \item ¿Cuáles son los pasos de cada tarea?
    \item ¿Cuáles son el resultado y el producto de cada tarea?
    \item ¿De dónde proviene la información de cada tarea?
    \item ¿Cómo se utiliza la información que resulta de cada tarea?
    \item ¿Qué herramientas se utilizan para realizar cada tarea?
    \item ¿Qué problemas tienen las personas para realizar cada tarea? ¿Qué tipo de errores son comunes? ¿Qué los causa? ¿Qué tan dañinos son los errores?
    \end{itemize}
\end{itemize}
\end{frame}

\begin{frame}
\frametitle{Johnson (2007)}
\begin{itemize}
\color{red}
\item Análisis de Tareas
    \par
    \justify
    \color{black}
    Una vez que se responden estas preguntas (observando y / o entrevistando a las personas que hacen las tareas que la herramienta admitirá), el siguiente paso es comenzar a dibujar posibles interfaces de usuario. El siguiente paso es diseñar un modelo conceptual para la herramienta que se centra en las tareas y objetivos de los usuarios (Johnson & Henderson, 2002). Después de haber diseñado un modelo conceptual centrado en tareas, tan simple como sea posible, y lo más consistente posible, puede diseñar una interfaz de usuario que minimiza el tiempo y la experiencia necesarios para utilizar la aplicación para convertirse en un proceso automático.
    \end{itemize}
    
\end{frame}

\begin{frame}
\frametitle{Johnson (2007)}
A continuación, hablaremos un poco sobre los principios básicos de la interaccion Persona-ordenador.
\begin{itemize}
\color{red}
\item Principio 1: Centrarse en los usuarios y sus tareas, no en la tecnología
    \par
    \justify
    \color{black}
    \begin{itemize}
    \item Entender a los usuarios
    \item Entender las tareas
    \item Considere el contexto en el que funcionará el software
    \end{itemize}
    \par
    Para un correcto cumplimiento de los objetivos, la tecnología toma un lugar secundario y se prioriza el entendimiento del usuario y la eficacia del sistema.
\end{itemize}
\end{frame}

\begin{frame}
\frametitle{Johnson (2007)}
\begin{itemize}
\color{red}
\item Principio 2: Considere la función primero, la presentación después
    \par
    \justify
    \color{black}
    \begin{itemize}
    \item Desarrollar un modelo conceptual que cumple de manera completa el objetivo del mismo.
    \end{itemize}
     \par
    El cumplimiento de la tarea siempre será la prioridad.
\end{itemize}
\end{frame}

\begin{frame}
\frametitle{Johnson (2007)}
\begin{itemize}
\color{red}
\item Principio 3: Conforme a la visión de la tarea de los usuarios
    \par
    \justify
    \color{black}
    \begin{itemize}
    \item Lucha por la naturalidad
	\item Utilice el vocabulario de los usuarios, no el suyo
	\item Mantenga los componentes internos del programa dentro del programa
	\item Encuentre el punto correcto sobre el equilibrio entre potencia y complejidad
    \end{itemize}
    \par
    Para poder brindar algo optimo, el lenguaje a utilizar para los usuarios debe ser lo más simple posible, no resultar forzado para ellos, y debe ser sencilla y práctica.
\end{itemize}
\end{frame}

\begin{frame}
\frametitle{Johnson (2007)}
\begin{itemize}
\color{red}
\item Principio 4: Diseño para el caso común
    \par
    \justify
    \color{black}
    \begin{itemize}
    \item Hacer que los resultados comunes sean fáciles de lograr
	\item Dos tipos de "comunes": "cuántos usuarios" y "con qué frecuencia"
	\item Diseño para casos básicos; no te preocupes por los casos de "borde"
    \end{itemize}
    \par
     Como ya se mencionó, la practicidad y el nivel de interacción deben ser simples y prácticos para el entendimiento e interacción del usuario.
\end{itemize}
\end{frame}

\begin{frame}
\frametitle{Johnson (2007)}
\begin{itemize}
\color{red}
\item Principio 5: No complique la tarea de los usuarios
    \par
    \justify
    \color{black}
    \begin{itemize}
    \item No les dé problemas adicionales a los usuarios
	\item No hagas que los usuarios razonen por eliminación
    \end{itemize}
    \par
    Para el usuario, debe ser sencillo interactuar con el sistema, por esa razón las tareas que se le proporciona deben ser claras y exactas, no interponer varias acciones o tareas que puedan confundir a los usuarios.
\end{itemize}
\end{frame}

\begin{frame}
\frametitle{Johnson (2007)}
\begin{itemize}
\color{red}
\item Principio 6: Facilitar el aprendizaje
    \par
    \justify
    \color{black}
    \begin{itemize}
    \item Piense "de afuera hacia adentro", no "de adentro hacia afuera"
	\item Consistencia, consistencia, consistencia 
	\item Proporcionar un entorno de bajo riesgo
    \end{itemize}
    \par
    Para el que interacciona, las tareas asignadas deben ser simples, que no comprometan la funcionalidad del sistema.
\end{itemize}
\end{frame}

\begin{frame}
\frametitle{Johnson (2007)}
\begin{itemize}
\color{red}
\item Principio 7: Entregue información, no solo datos
    \par
    \justify
    \color{black}
    \begin{itemize}
    \item Diseñe las exhibiciones cuidadosamente; conseguir ayuda profesional
	\item La pantalla pertenece al usuario 
	\item Preservar la inercia de la pantalla 
    \end{itemize}
\end{itemize}
\end{frame}

\begin{frame}
\frametitle{Johnson (2007)}
\begin{itemize}
\color{red}
\item Principio 8: Diseño para la capacidad de respuesta
    \par
    \justify
    \color{black}
    \begin{itemize}
    \item Reconocer las acciones del usuario al instante 
	\item Informar a los usuarios cuando el software está ocupado y cuando no 
	\item Libera a los usuarios para que hagan otras cosas mientras esperan 
	\item Animar el movimiento de forma suave y clara 
	\item Permitir a los usuarios abortar operaciones prolongadas que no desean
	\item Permitir a los usuarios estimar cuánto tiempo tomarán las operaciones
	\item Trate de permitir que los usuarios establezcan su propio ritmo de trabajo
    \end{itemize}
    \par
    Para el usuario, las acciones deben ser comprensibles, para esto, la información que se le brinda debe estar a un nivel simple de entendimiento, con una retroalimentación moderada, y poder darle la opción de cancelar y medir el proceso de trabajo.
\end{itemize}
\end{frame}

\begin{frame}
\frametitle{Johnson (2007)}
\begin{itemize}
\color{red}
\item Principio 9: Pruébelo con los usuarios; entonces arréglalo
    \par
    \justify
    \color{black}
    \begin{itemize}
    \item Los resultados de las pruebas pueden sorprender incluso a los         diseñadores experimentados 
	\item Programe tiempo para corregir los problemas encontrados por las pruebas
	\item Las pruebas tienen dos objetivos: informativos y sociales.
	\item Hay pruebas para cada momento y propósito
    \end{itemize}
\end{itemize}
\end{frame}

\begin{frame}
\frametitle{Johnson (2007)}
\begin{itemize}
\color{red}
\item ¿Cuánto tiempo le toma a nuestro cerebro…?
\par
\color{black}
A continuación, se enumeran las duraciones medias perceptuales y algunas funciones cognitivas del cerebro que afectan nuestras percepciones del sistema sensitivo. Los tiempos se enumeran del más corto al más largo:
    \par
    \justify
    \color{black}
    \begin{itemize}
    \item 0.001 segundos: La brecha de silencio más corta que podemos detectar en un sonido.
    \item 0.002 segundos: Tiempo mínimo entre los picos de las neuronas auditivas, las más rápidas en el cerebro.
    \item 0.005 segundos: El tiempo más corto en el que se puede mostrar un estímulo visual y que aún nos afecta, tal vez de forma inconsciente.
    \item 0.01 segundos: Mínimo retraso notable en la tinta, como cuando alguien dibuja con un lápiz.
    \item 0.02 segundos: Intervalo máximo para la fusión auditiva de pulsos de sonido sucesivos en un tono agudo.
    \item 0.05 segundos: Intervalo máximo para la fusión visual de imágenes sucesivas.
    \end{itemize}
\end{itemize}
\end{frame}

\begin{frame}
\frametitle{Johnson (2007)}
\begin{itemize}
\par
    \par
    \justify
    \color{black}
    \begin{itemize}
    \item 0.08 segundos: Velocidad del reflejo de estremecimiento (respuesta motora involuntaria a un posible peligro).
    \item  0.1 segundos: Lapso de tiempo entre un evento visual y nuestra percepción de él.
    \item 0.1 segundos: Duración de la sacádica (movimiento involuntario del ojo), durante la cuál la visión es suprimida.
    \item 0.14 segundos: Intervalo máximo entre eventos para la percepción de que un evento causó otro evento.
    \item 0.15 segundos: Tiempo necesario para que el cerebro de un lector experto comprenda una palabra impresa.
    \item 0.2 segundos: Tiempo de subitizar hasta cuatro o cinco elementos en nuestro campo visual. 
    \item 0.25 segundos: Tiempo en el que se identifica (o sea, saber qué es) un objeto visto.
    \item 0.3 segundos: Tiempo necesario para contar mentalmente cada elemento en una escena cuando hay más de cuatro elementos.
    
    \end{itemize}
\end{itemize}
\end{frame}

\begin{frame}
\frametitle{Johnson (2007)}
\begin{itemize}
\par
    \par
    \justify
    \color{black}
    \begin{itemize}
    \item 0.5 segundos: "Parpadeo" de atención después del reconocimiento de un objeto.
    \item 0.7 segundos: Tiempo de reacción visomotor (respuesta intencional a un evento inesperado).
    \item Un segundo: Duración máxima del espacio silencioso entre los turnados en la conversación de persona a persona.
    \item 6 - 30 segundos: Duración de la atención ininterrumpida a una sola tarea.
    \item 1 - 5 minutos: Tiempo en el que se toman decisiones en casos de emergencia.
    \item 1 - 10 días: Duración de la decisión de una compra importante.
    \item 20 años: Tiempo en el que se escoge una carrera para toda la vida.

    \end{itemize}
    
\end{itemize}
\end{frame}

\section{References}
%References frame
\begin{frame}
\frametitle{References}
\begin{itemize}
\item Jhonson J. (2014). Designing with the Mind in mind. 2nd. edition.
\end{itemize}
\end{frame}

\end{document}